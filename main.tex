%%%%%%%%%%%%%%%%%%%%%%%%% 
%% SITE A GARDER : http://titilog.free.fr/


\documentclass[10pt]{beamer}
%\includeonlyframes{yy}
\usepackage[french]{babel}
\usepackage{color, colortbl}
\definecolor{Gray}{gray}{0.9}
\newcolumntype{g}{>{\columncolor{Gray}}c}
\hypersetup{%
  % pdfborder = {0 0 0},
  colorlinks, urlcolor=blue, linkcolor=, }
  
\makeatletter \let\@mycite\@cite
\def\@cite#1#2{{\hypersetup{linkcolor=mLightBlue}[{#1\if@tempswa ,
      #2\fi}]}} \makeatother

\usetheme{m} %%%%%
% \usepackage[version=3]{mhchem}
\usepackage{caption}
\captionsetup[figure]{labelformat=empty}
\usepackage{fontspec}
% \usepackage[backend=bibtex]{biblatex} \usepackage[square]{natbib}
% \bibliography{main}
\usepackage[french]{algorithm2e}
\usefonttheme[onlymath]{serif} \usepackage{amsmath}
\usepackage{subcaption} \usepackage{appendixnumberbeamer}
\usepackage{tabularx, booktabs} \usepackage{multirow}
\usepackage{multicol} \usepackage{array} \usepackage{pbox}
\PassOptionsToPackage{enumerate}{shortlabels} \newcommand\ExtraSep
{\dimexpr\cmidrulewidth\relax}
\captionsetup{font=scriptsize,labelfont=scriptsize}
% \addtobeamertemplate{background canvas}{\transfade[duration=0.05]}{}
\title{Problématiques en Informatique soulevées par le projet Optimum}

\author{{\large Florent \textsc{Grélard}}\\\\
  Serge \textsc{Miguet}, Mihaela \textsc{Scuturici}, Mehdi
  \textsc{Ayadi} }

\titlegraphic{\vspace{-1cm}\hspace*{0.12\linewidth}%
   \includegraphics[width=0.25\linewidth]{fig/logo_liris2}\hspace*{0.1\textwidth}~%
   \includegraphics[width=0.1\linewidth]{fig/logo_lyon2}\hspace*{0.12\linewidth}%
   \includegraphics[width=0.25\linewidth]{fig/logo_imu}\hspace*{0.12\linewidth}%
 }
\setbeamercolor{bbb}{fg=red}

\let\oldfootnotesize\footnotesize
\renewcommand*{\footnotesize}{\oldfootnotesize\tiny}
\newcommand\labelitemi{$\bullet$}
\renewcommand{\thefootnote}{[\arabic{footnote}]}
% \newrobustcmd*{\footlessfullcite}{\AtNextCite{\renewbibmacro{in:}{}\renewbibmacro{year:}{}}\footfullcite}
% \newrobustcmd*{\lessfullcite}{\AtNextCite{\renewbibmacro{in:}{}\renewbibmacro{year:}{}}\fullcite}



\newcommand{\backupbegin}{ \newcounter{framenumberappendix}
  \setcounter{framenumberappendix}{\value{framenumber}} }
\newcommand{\backupend}{
  \addtocounter{framenumberappendix}{-\value{framenumber}}
  \addtocounter{framenumber}{\value{framenumberappendix}} }
% http://mcclinews.free.fr/latex/introbeamer/les_couleurs.html
\begin{document}

% affiche le logo en bas à droite
\logo{\includegraphics[height=0.5cm]{fig/logo_liris.png}}
% enlève la barre de navigation
\setbeamertemplate{navigation symbols}{ }
\setbeamertemplate{blocks}[rounded][shadow=false]
% Ombre aux blocks
\setbeamertemplate{caption}{\insertcaption}

\date{27 juin 2018} % set the Date
\AtBeginSection[]{
  \begingroup \setbeamercolor{background canvas}{bg=mLightBlue}

  \setbeamercolor{background canvas}{bg=mLightBlue}

  \setbeamertemplate{subsection in toc}
  {\leavevmode\leftskip=2em$\bullet$\hskip1em\inserttocsubsection\par}

   \setbeamercolor{section in toc shaded}{bg=paleGrey}


  \setbeamercolor{section in toc}{fg=paleGrey, bg=mLightBlue}
  \setbeamercolor{local structure}{fg=mLightBlueLighter,bg=mLightBlue}
  \setbeamercolor{section in toc shaded}{fg=mLightBlue, bg=mLightBlue}
  \setbeamertemplate{section in toc}[circle] \setbeamertemplate{section
    in toc shaded}[default][80]


  \setbeamercolor{subsection in toc}{fg=paleGrey, bg=mLightBlue}
  \begin{frame}[plain]
    \frametitle{\textcolor{paleGrey}{Table des matières}}
    \tableofcontents[currentsection, hideothersubsections]
  \end{frame}
  \endgroup
} \renewcommand{\insertnavigation}[1]{}

\begin{frame}[plain]
  \titlepage
\end{frame}


\section{Enjeux}
\subsection{Passage à l'échelle}

\begin{frame}{Passage à l'échelle : exemple}
  Pour une requête donnée, l'\textbf{ensemble des photographies} est parcouru. \\[0.5cm]

  \begin{block}{Problème}
    Le temps de calcul \textbf{augmente proportionnellement} avec le nombre de photographies.
  \end{block}

  \begin{table}[]
    \centering
    \caption{Temps de réponse (en millisecondes) pour une recherche exhaustive :}
    \label{table:exhaustif}
    \begin{tabular}{cc}
      \hline
      Nombre de polygones & Exhaustive \\ \hline
      500                 & 4          \\
      5000                & 19         \\
      50000               & 220        \\
      500000               & 5200       \\ \hline
    \end{tabular}
  \end{table}

\end{frame}

\begin{frame}{Passage à l'échelle : exemple}
  \alert{R-Tree} : indexation des données spatiales dans des \textbf{rectangles imbriqués}. \\
  Permet de rejeter rapidement des zones de l'espace lors d'une recherche

    \begin{figure}
    \includegraphics<1>[width=0.45\textwidth]{fig/rtree_0}
    \includegraphics<2>[width=0.45\textwidth]{fig/rtree_1}
    \includegraphics<3>[width=0.45\textwidth]{fig/rtree_2}
    \includegraphics<4->[width=0.45\textwidth]{fig/rtree_4}
  \end{figure}
\end{frame}

\begin{frame}{Passage à l'échelle}
   \begin{table}[]
    \centering
    \caption{Temps de réponse (en millisecondes) pour une recherche exhaustive et pour une recherche sur un R-tree :}
    \label{my-label}
    \begin{tabular}{ccc}
      \hline
      Nombre de polygones & Exhaustive & R-tree \\ \hline
      500                 & 4          & 1      \\
      5000                & 19         & 1      \\
      50000               & 220        & 2      \\
      500000               & 5200       & 2      \\ \hline
    \end{tabular}
  \end{table}

  $\Rightarrow$ De façon générale, utiliser des structures de données et algorithmes \textbf{efficaces en temps de calcul}.
\end{frame}

\subsection{Correction de la position}
\begin{frame}{Correction de la position}
   \begin{block}{Problème}
        Le système GPS de l'appareil photo est source d'\textbf{imprécisions}.
      \end{block}
  \begin{columns}
    \begin{column}[c]{0.6\textwidth}
     

      \textbf{Correction de la position :}
      \begin{enumerate}
      \item<1-> Obtention d'une image de synthèse à partir de la position imprécise et d'un modèle 3D texturé de la ville
      \item<2-> Appariement de l'image de synthèse avec l'image réelle
      \item<3-> Correction de la position et de la direction (exemple : algorithme des huit points~\cite{Longuet1981})
      \end{enumerate}

    \end{column}
    \begin{column}[c]{0.4\textwidth}
      \begin{figure}[ht]
        \centering
        \includegraphics<1>[width=0.95\textwidth]{fig/recalage_0}
        \includegraphics<2>[width=0.95\textwidth]{fig/recalage_1}
        \includegraphics<3>[width=0.95\textwidth]{fig/recalage_2}
        \caption{}
        \label{}
      \end{figure}

    \end{column}
  \end{columns}
 
\end{frame}

\subsection{Appariement d'images}
\begin{frame}{Analyse du contenu de l'image}
  \textbf{Objectif :} analyser le contenu des images afin de faire des grappes plus pertinentes.

  \textbf{Pistes :}
  \begin{enumerate}
  \item Descripteurs de texture
  \item Segmentation sémantique : catégorise des régions de l'image en fonction du sens qu'on leur porte
  \end{enumerate}

\end{frame}

\section{Questions ouvertes}
\subsection{Outil}
\begin{frame}{Sur l'outil}
  Il est possible de générer un modèle 3D à partir d'images visualisant le même bâtiment. \\
  \alert{Est-ce que cela peut être pertinent ?} \\[1cm]

  Images d'archives (Fonds Paul) :
  \alert{Comment les intégrer et les représenter ?}\\[1cm]
  
  \alert{D'autres suggestions...?}
\end{frame}

\subsection{Site Web du projet}
\begin{frame}{Sur le site Web du projet}
  Présenter les questionnements et réflexions. \\
  \alert{Publications périodiques, ou bien unique à l'issue du projet ?} \\[1cm]

  Présenter les photographies de Guillaume. \\
  \alert{Quelle sélection opérer?}
\end{frame}

 \begin{frame}[allowframebreaks]
  \frametitle{Références}
   \setbeamertemplate{bibliography item}{$\bullet$}
  \bibliographystyle{apalike}
   \scriptsize{
     \bibliography{main}
   }
 \end{frame}

 


\end{document}

%%% Local Variables:
%%% mode: latex
%%% TeX-master: t
%%% End:
